% Preamble
\documentclass{paper}

\newcommand{\thetitle}{The Impact of Boluan Fanzheng on Student Protests in Dengist China}
\newcommand{\researchquestion}{How was the 1989 Tiananmen Square Protests a result of Deng Xiaoping's political reforms?}
\newcommand{\wordcount}{1619}

\usepackage{XCharter}

% Page Numbering
\usepackage{fancyhdr}
\pagestyle{fancy}
\fancyhf{}
\fancyhead[R]{\thepage}
\setlength{\headheight}{15pt}

% Bibliography
\usepackage[backend=biber, notes, sorting=cms, legalnotes=false]{biblatex-chicago}
\usepackage{fnpct}
\addbibresource{bibliography.bib}
\usepackage[bottom]{footmisc}
\usepackage{bookmark}

% Quotes
\usepackage[english]{babel}
\usepackage[autostyle, english=american]{csquotes}
\MakeOuterQuote{"}

% Justifying
\usepackage{ragged2e}

% Paragraph skip
\usepackage[parfill]{parskip}

\begin{document}
\insertTitlePage
\tableofcontents
\thispagestyle{empty}
\newpage
\setcounter{page}{1}
\pagenumbering{arabic}
\doublespacing
\justifying

%Intro
\section{Introduction}
The 20th century was a time of turmoil for the People's Republic of China.
From the Communist Revolution to the Cultural Revolution, the Chinese government has had huge shifts in leadership and policy over its relatively short lifespan.
Two of the most significant upheavals in modern Chinese history were the shift from an idealized Maoist economy and political structure to an economic powerhouse, and the political and civil unrest surrounding the 1989 Tiananmen Square Protests and Massacre.
Despite seeming unrelated, these two events were highly linked by a chain of causation, and display the way in which party leadership and the tide of political power massively impacted policy in Dengist China.
This investigation will explore the question: How was the 1989 Tiananmen Square Protests and Massacre a result of Deng Xiaoping's political reforms?
The investigation will focus on the years from 1977 to 1989, from Deng Xiaoping's proposal of \emph{Boluan Fanzheng} at the Third Plenum of the 11th Central Committee in 1978 to the aftermath of the massacre on the sixth of June 1989.
The investigation will accomplish this through an analysis of the political events that led to the Protests and Massacre and the tides of political power within the Communist Party and Chinese Government with context provided by two primary sources.

\section{Identification and Evaluation of Sources}
The first source to be evaluated is Zhang Liang's \textit{The Tiananmen Papers},\autocite{tiananmen} a translated collection of secret Chinese documents surrounding the June 4th massacre published in 2001.
This content of the source is valuable because it shows the situation in the Chinese Communist Party at the time of the massacre and the internal politics surrounding the decision to impose martial law in the capital.
The purpose of the document was to reveal the internal party correspondence surrounding the June 4th incident, and bring to light some of the more political aspects of it within the Chinese Communist Party.
The source is limited by the author's choice to publish under a pseudonym to protect themselves from persecution.
The use of a pseudonym makes the documents within more dubious in origin and opens up questions to their legitimacy.
As a result, of the documents have been questioned by peers and is denied by the Chinese government---who has a vested interest in obscuring the events---and banned in mainland China.

The second source to be evaluated is \textit{On the Reform of the System of Party and State Leadership},\autocite{reforms} a speech given by Deng Xiaoping before a full meeting of the CCP Politburo that launched political reforms in China.
The origin of the source is valuable because it is a primary source that outlines the steps that will take place and a lot of the justification behind the reforms.
The purpose of the speech was "to take full advantage of the superiority of socialism and speed up China's modernization" The contents of the speech is a description of the steps to take to kick-start a separation between party and government in China and move towards a more open and prosperous nation, with less focus on ideological purity and a focus on rapid and effective industrialization and modernization through a more free market economy.
One of the limitations of the speech is that it was given to a very specific audience with political intent, and as such may not necessarily represent Xiaoping's actual beliefs on the topic or the policies that he would put into place following the speech.
At the same time, this limitation can provide value to our investigation, as the intent with which the speech is written is indicative of the political environment in which Xiaoping instituted his reform.
%
\newpage
%
\section{Investigation}
In September 1977, after the conclusion of the Cultural Revolution with the arrests of the Gang of Four, Deng Xiaoping, the new \emph{de facto} paramount leader of China, launched a series of political and economic reforms called Boluan Fanzheng, literally "eliminating chaos and returning to normal."
The reason for these reforms was China's poor economic growth of the 1970s, which was far outstripped by nearby capitalist nations such as South Korea and the Republic of China (commonly known as Taiwan).
Seizing the opportunity following the death of Mao Zedong, Deng Xiaoping and other communist party reformers instituted the many political and economic policy changes that made up Boluan Fanzheng.
The reforms began the process of bringing Communist China from a state focused on ideological purity and Maoist Principles away from the turmoil of the Cultural Revolution and towards becoming a global economic powerhouse.
These reforms began with the dissolution of the communes of Maoist China, reverting agriculture back to privately managed farmland, and the allowance of entrepreneurs to start private enterprises.\autocite{reforms}
These provisions were a clear step away from the ideas of communism that the nation was founded upon, and would be the first step towards a movement that would shake the social and political fabric of 1980s China.

Deng Xiaoping signaled the beginning of his reforms in the speech \textit{On the Reform of the System of Party and State Leadership} in 1977, which outlined the need for the prioritization of economic growth and the division between the party and state.
Xiaoping's speech sparked a series of reforms in both the economic sector and the political landscape of China as reformist politicians began the process of loosening state control on the private sector, lifting price controls, opening up to foreign investment, and encouraging criticism of the Cultural Revolution and Maoist China.
Some of the most important reformists under Xiaoping were Hu Yaobang, at the time the General Secretary of the Chinese Communist Party, and Zhao Ziyang, the Premier, but they faced significant resistance from the primarily conservative \emph{Eight Elders} of the Chinese Communist Party, an informal group of party elders with substantial sway in the government.
Despite the resistance, Yaobang and Ziyang managed to make significant progress and became popular figures with the progressive youth.

Despite their status among the youth, Ziyang and Yaobang were unpopular with much of the conservative members of the communist party and Chinese government, who viewed their political reforms as \emph{bourgeois liberalization} and a perversion of socialism.
After his lackadaisical response to the 1986--1987 democratic student protests, conservative party members seized on the opportunity to attack Yaobang.
They claimed that his response was too progressive and constituted bourgeois liberalization.\autocite{deng}
As a result of these claims, Yaobang was disgraced and forced to resign from his role as General Secretary (leader of the Communist Party).\autocite{calhoun}
In his stead, Ziyang took the position of General Secretary and was replaced as premier (head of government) by conservative Li Peng.

After Yaobang's death two years later, youth began demanding that the government honor his contributions to the Party and China as a whole.
These demands quickly developed into large scale protests in and around the Tiananmen Square in Beijing, with many protesters believing his sudden death was related to his forced resignation.\autocite{calhoun}
Yaobang was a loved figure among the progressive youth, and over 100,000 students marched on the square, quickly developing the small-scale demands into massive protest.
Quickly, protesting students began to make several demands of the government, such as affirmations of Yaobang's political stance and legacy, the cessation of press censorship, and further progress of political and economic progress.

In reaction to these protests, hard line conservative premier Li Peng and the Eight Elders managed to convince Xiaoping to allow them to mobilize the People's Liberation Army to clear the protests from the square.
This would escalate the protests into a massacre as the government's response to the protests became violent.
Zhao Ziyang, as a supporter of these protests, was expelled from the party and placed under house arrest.\autocite{tiananmen}
In the aftermath of the protests, the progressive movement spearheaded by Ziyang and Yaobang was slowed significantly, and conservative members wielded significantly more power in the party and government.

The Tiananmen Square Protests were a result of dissatisfaction with the pace of political change in China in the 70s and 80s, and a reaction to the political turmoil that drove Hu Yaobang out of the party.
The Massacre was similarly a reaction to the progressive reforms by the hard line conservative faction within the Communist Party and government.
Had the political reforms of Boluan Fanzheng never occurred, and had China remained as it was at the time of Mao Zedong's death, it is unlikely that such a protest and massacre would have occurred, but because of the unique political climate and divisions in the party created by Xiaoping's political reforms, one of the most significant massacres of the late 20th century occurred.
%
\newpage
%
\section{Reflection}
This investigation helped me understand the methods historians use by utilizing them in my research.
Through this investigation, I have managed to develop my skills in analyzing sources and constructing an understanding through comparing and contrasting sources, both of which are important tools for historians to investigate events of the past.
This investigation was also my first time actually purchasing physical sources for my research, which was a unique experience compared to my papers in the past, in which I have utilized mostly digital sources that made it easy to search for helpful information through keywords.
This forced me to read through the source more completely instead of just the most important parts, leading to a better understanding of the topic.

In history class we typically learn the "accepted" version of events and typically do not have to worry about conflicting arguments and sources outside of practice IB exams and essays, but with this investigation I had to deal with conflicting sources and understand the biases and contexts in which the sources were written in order to create an understanding of the Tiananmen Square Protests and Massacre.

\singlespacing
\newpage
\printbibliography
\thispagestyle{empty}
\end{document}
