% Preamble
\documentclass{paper}

\newcommand{\thetitle}{The Impact of Boluan Fanzheng on Student Protests in Dengist China}
\newcommand{\researchquestion}{How was the 1989 Tiananmen Square Protests a result of Deng Xiaoping's political reforms?}
\newcommand{\wordcount}{1120}
\usepackage[backend=biber, sorting=none]{biblatex}
\addbibresource{bibliography.bib}

\begin{document}
\pagenumbering{gobble}
\insertTitlePage
\tableofcontents
\newpage
\setcounter{page}{1}
\pagenumbering{arabic}
\doublespacing

%Intro
\section{Introduction}
This investigation will explore the question: How was the 1989 Tiananmen Square Protests a result of Deng Xiaoping's political reforms? The investigation will focus on the years from 1977 to 1989, from Deng Xiaoping's proposal of Boluan Fanzheng at the Third Plenum of the 11th Central Committee in 1978 to the aftermath of the massacre on the sixth of June 1989.

\section{Identification and Evaluation of Sources}
The first source to be evaluated is Zhang Liang's \textit{The Tiananmen Papers} \cite{tiananmen}, a translated collection of secret Chinese documents surrounding the June 4th massacre published in 2001. This content of the source is valuable because it shows the situation in the Chinese Communist Party at the time of the massacre and the internal politics surrounding the decision to impose martial law in the capital. The purpose of the document was to reveal the internal party correspondence surrounding the June 4th incident, and bring to light some of the more political aspects of it within the Chinese Communist Party. The source is limited by the author's choice to publish under a pseudonym to protect themselves from persecution. The legitimacy of the documents has been questioned by peers and is denied by the Chinese government.

The second source to be evaluated is \textit{On the Reform of the System of Party and State Leadership} \cite{reforms}, a speech given by Deng Xiaopeng before a full meeting of the CCP Politburo that launched political reforms in China. The origin of the source is valuable because it is a primary source that outlines the steps that will take place and a lot of the justification behind the reforms. The purpose of the speech was 'to take full advantage of the superiority of socialism and speed up China's modernization' The contents of the speech is a description of the steps to take to kickstart a separation between party and government in China and move towards a more open and prosperous nation, with less focus on ideological purity and a focus on rapid and effective industrialization and modernization through a more free market economy.

\section{Investigation}
In September of 1977, after the conclusion of the Cultural Revolution with the arrests of the Gang of Four, Deng Xiaopeng, the new paramount leader of China, launched a series of political and economic reforms called 'Boluan Fanzheng,' literally 'eliminating chaos and returning to normal' that began the process of bringing Communist China from a state focused on ideological purity and Maoist Principles away from the turmoil of the Cultural Revolution and towards becoming a global economic powerhouse. These reforms began with the dissolution of the communes of Maoist China, reverting agriculture back to privately managed farmland, and the allowance of entrepreneurs to start private enterprises \cite{reforms}. These provisions were a clear step away from the ideas of communism that the nation was founded upon, and would be the first step towards a movement that would shake the social and political fabric of 1980s China.

Xiaopeng's speech in 1977 sparked a series of reforms in both the economic sector and the political landscape of China as reformist politicians began the process of loosening state control on the private sector, lifting price controls, opening up to foreign investment, and encouraging criticism of the Cultural Revolution and Maoist China. Some of the most important reformists were Hu Yaobang, at the time the General Secretary of the Chinese Communist Party and Zhao Ziyang, the Premier at the time, but they faced resistance primarily from the primarily conservative Eight Elders of the Chinese Communist Party, an informal group of party elders with substantial sway in the government. After what conservatives viewed as a lacking response to the 1986-1987 student protests that called for the right to nominate candidates for congress and Hu's refusal to dismiss protest leaders from the party, they managed to oust Yaobang from party leadership citing his reforms as being too liberal, calling them 'bourgeois liberalization' \cite{deng}. The position of General Secretary---leader of the Chinese Communist Party---was transferred to then Premier Zhao Ziyang, but the position of Premier---the leader of the government---was given to the conservative Li Peng.

Two years later, on the day of Yaobang's death, thousands of students began to gather in and around Tiananmen Square in mourning of the former General Secretary, many of them believing his sudden death was related to his forced resignation two years prior \cite{calhoun}. Quickly, posters began to appear calling to speed up reforms to counter corruption, move towards democracy, implement the freedom of the press, and implement various other demands. The protesting students hoped to force the government to accede to a list of seven demands, including but not limited to: affirming Yaobang's views on democracy and freedom, ceasing press censorship, and admitting that the government's campaigns against bourgeois liberalization and western ideas were wrong, showing that at the protests' base was the political reforms---and the party's resistance to them---that Deng Xiaopeng kickstarted at the end of the Cultural Revolution.

\newpage

\section{Reflection}
This investigation helped me understand the methods historians use by utilizing them in my research. Through this investigation I have managed to develop my skills in analyzing sources and constructing an understanding through comparing and contrasting sources, both of which are important tools for historians to investigate events of the past. This investigation was also my first time actually purchasing physical sources for my research, which was a unique experience compared to my papers in the past, in which I have utilized mostly digital sources that made it easy to search for helpful information through keywords. This forced me to read through the source more completely instead of just the most important parts, leading to a better understanding of the topic.

In history class we typically learn the 'accepted' version of events and typically do not have to worry about conflicting arguments and sources outside of practice IB exams and essays, but with this investigation I had to deal with conflicting sources and understand the biases and contexts in which the sources were written in order to create an understanding of the Tiananmen Square Protests and Massacre.

\singlespacing
\newpage
\pagenumbering{gobble}
\printbibliography
\end{document}
