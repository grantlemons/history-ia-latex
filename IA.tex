% Preamble
\documentclass{paper}

\newcommand{\thetitle}{The Impact of Boluan Fanzheng on Student Protests in Dengist China}
\newcommand{\researchquestion}{To what extent were the 1989 Tiananmen Square Protests a result of Deng Xiaoping’s political reforms?}
\newcommand{\wordcount}{923}

\begin{document}
\insertTitlePage

%Intro
\section*{Introduction}
\begin{doublespace}
This investigation will explore the question: To what extent were the 1989 Tiananmen Square Protests a result of Deng Xiaoping's political reforms? The investigation will focus on the years from 1977 to 1989, from Deng Xiaoping’s proposal of Boluan Fanzheng at the Third Plenum of the 11th Central Committee in 1978 to the aftermath of the massacre on the sixth of June 1989.
\end{doublespace}

\section{Identification and Evaluation of Sources}
\begin{doublespace}
\subfile{sections/source1opcvl.tex}
\subfile{sections/source2opcvl.tex}
\end{doublespace}

\section{Investigation}
\begin{doublespace}
In September of 1977, after the conclusion of the Cultural Revolution with the arrests of the Gang of Four, Deng Xiopeng, the new paramount leader of China, launched a series of political and economic reforms called ‘Boluan Fanzheng,’ literally ‘eliminating chaos and returning to normal’ that began the process of bringing Communist China from a state focused on ideological purity and Maoist Principles away from the turmoil of the Cultural Revolution and towards becoming a global economic powerhouse. These reforms began with the dissolution of the communes of Maoist China, reverting agriculture back to privately managed farmland, and the allowance of entrepreneurs to start private enterprises. These provisions were a clear step away from the ideas of communism that the nation was founded upon, and would be the first step towards a movement that would shake the social and political fabric of 1980s China.

Xiaopeng’s speech in 1977 sparked a series of reforms in both the economic sector and the political landscape of China as reformist politicians began the process of loosening state control on the private sector, lifting price controls, opening up to foreign investment, and encouraging criticism of the Cultural Revolution and Maoist China. Some of the most important reformists were Hu Yaobang, at the time the General Secretary of the Chinese Communist Party and Zhao Ziyang, the Premier at the time, but they faced resistance primarily from the primarily conservative Eight Elders of the Chinese Communist Party, an informal group of party elders with substantial sway in the government. After what conservatives viewed as a lacking response to the 1986-1987 student protests that called for the right to nominate candidates for congress and Hu’s refusal to dismiss protest leaders from the party, they managed to oust Yaobang from party leadership citing his reforms as being too liberal, calling them ‘bourgeois liberalization’. The position of General Secretary - leader of the Chinese Communist Party - was transferred to then Premier Zhao Ziyang, but the position of Premier - the leader of the government - was given to the conservative Li Peng.

Two years later, on the day of Yaobang’s death, thousands of students began to gather in and around Tiananmen Square in mourning of the former General Secretary, many of them believing his sudden death was related to his forced resignation two years prior (Calhoun, 1989). Quickly, posters began to appear calling to speed up reforms to counter corruption, move towards democracy, implement the freedom of the press, and implement various other demands. The protesting students hoped to force the government to accede to a list of seven demands, including but not limited to: affirming Yaobang’s views on democracy and freedom, ceasing press censorship, and admitting that the government’s campaigns against bourgeois liberalization and western ideas were wrong, showing that at the protest’s base were the political reforms - and the party’s resistance to them - that Deng Xiaopeng kickstarted at the end of the Cultural Revolution.
\end{doublespace}

(I think I ended up having too much of a ‘story’ and nowhere near enough of an ‘argument,’ I’ll also lengthen it and add my citations in the coming weeks)\\
(I also only really argued one side, I might need to look back over my research question and see if I can change it to better fit my research, as I can’t find anything saying that the reforms weren’t the main source of the protests - at least that can be a good thing to discuss in my reflection)\\
(If I can find a counterargument I’ll be able to lengthen it significantly)\\
\newpage

\section{Reflection}
\begin{doublespace}
\end{doublespace}

\end{document}












