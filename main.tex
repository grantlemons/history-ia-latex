% Preamble
\documentclass[letterpaper, 12pt]{article}

% Paragraph jumps and indentation
\setlength{\parskip}{1.6em}
\setlength{\parindent}{1.25cm}

% Border
\usepackage[includefoot, left=1in, right=1in, top=1in, bottom=0.75in]{geometry}

% Footer
\usepackage{fancyhdr}
\fancyhf{}
\rfoot{\textit{\thepage}}
\renewcommand{\headrulewidth}{0.0pt}
\renewcommand{\footrulewidth}{0.0pt}
\pagestyle{fancy}

% Packages
\usepackage[dvipsnames]{xcolor}
\usepackage{mathtools}
\usepackage{amsmath}
\usepackage{amsfonts}
\usepackage{siunitx}
\usepackage[english]{babel}
\usepackage{microtype}

% Spacing
\usepackage{setspace}
\doublespacing
\usepackage{titlesec}
\titlespacing{\section}{0pt}{6pt plus 2pt minus 2pt}{0pt plus 2pt minus 2pt}
\titlespacing{\subsection}{0pt}{6pt plus 2pt minus 2pt}{0pt plus 2pt minus 2pt}
\titlespacing{\subsubsection}{0pt}{6pt plus 2pt minus 2pt}{0pt plus 2pt minus 2pt}

% Fonts
\usepackage{fontspec}
\setmainfont{Calibri}

% Images
\usepackage{graphicx}
\graphicspath{ {./images/} }
\usepackage{wrapfig}
\usepackage{float}

% Tables
\usepackage{multirow}
\usepackage{array}
\usepackage{tabu}
\titleformat{\section}
{\normalfont\large\bfseries}{\thesection}{1em}{}
\titleformat{\subsection}
{\normalfont\large\bfseries}{\thesubsection}{1em}{}

% Equation numbering
\counterwithin{equation}{section}
\usepackage[hidelinks]{hyperref}
\urlstyle{same}

% Document Information
\newcommand{\thetitle}{The Impact of Boluan Fanzheng on Student Protests in Dengist China}
\newcommand{\researchquestion}{To what extent was the Tiananmen Square Massacre a result of Deng Xiaoping's political reforms?}
\author{Grant Lemons}
\date{\today}

% Wordcount Setup
\newcommand{\wordcount}{0}%\input|"texcount -1 -sum */*.tex"}

% Subfile Setup
\usepackage{subfiles}

\begin{document}

% Title Page
\begin{titlepage}
\begin{center}
\vspace*{6.3cm}
\textbf{\Huge \thetitle}\\
\textit{\researchquestion}\\
\vspace{2.5cm}
\textit{Word count: \wordcount}\\
\vspace{0.6cm}
\textit{\today}\\

\end{center}
\end{titlepage}

% Table of Contents
\tableofcontents
\newpage

%Intro
This investigation will explore the question: To what extent was the Tiananmen Square Massacre a result of Deng Xiaoping's political reforms? The investigation will focus on the years from 1980 to 1989, from Deng Xiaoping’s speech \textit{On the Reform of the Party and State Leadership System} to the end of the massacre on the sixth of June 1989.

\section{Section 1: Identification and Evaluation of Sources}
\subfile{sections/source1opcvl.tex}
\subfile{sections/source2opcvl.tex}

\section{Section 2: Investigation}
In September of 1977, after the conclusion of the Cultural Revolution with the arrests of the Gang of Four, Deng Xiopeng, the new paramount leader of China, launched a series of political and economic reforms called ‘Boluan Fanzheng’, literally ‘bringing order out of chaos’ that began the process of bringing Communist China from a state focused on ideological purity and maoist principles to a global economic powerhouse. These reforms began with the dissolution of the communes of Maoist China, reverting agriculture back to privately managed farmland, and the allowance of entrepreneurs to start private enterprises. These provisions were a clear step away from the ideas of communism that the nation was founded upon, and would be the first step towards a movement that would shake the social and political fabric of 1980s China.
In 1989, after the death of former General Secretary Hu Yaobang, thousands of students began to gather in and around Tiananmen Square in mourning of the former General Secretary, many of them believing his sudden death was related to his forced resignation two years prior (Calhoun, 1989). Quickly, posters began to appear calling for reforms to counter corruption, move towards democracy, and implement the freedom of the press. Yaobang had been forced to resign due to his ‘excessive liberalization’ after his ‘lax’ response to the student protests of 1986 and 1987 and what Deng Xiaopeng called ‘bourgeois liberalization’ and had been a driving force in the political reforms and free market reforms of the Boluan Fanzheng campaign. The protesting students hoped to force the government to accede to their Seven Demands, including but not limited to: affirming Yaobang’s views on democracy and freedom, ceasing press censorship, and admit that the government’s campaigns against ‘bourgeois liberalization’ and western ideas.

\section{Section 3: Reflection}


\end{document}












